% ****************************************************************************************************
% classicthesis-config.tex 
% formerly known as loadpackages.sty, classicthesis-ldpkg.sty, and classicthesis-preamble.sty 
% Use it at the beginning of your ClassicThesis.tex, or as a LaTeX Preamble 
% in your ClassicThesis.{tex,lyx} with \input{classicthesis-config}
% ****************************************************************************************************  
% If you like the classicthesis, then I would appreciate a postcard. 
% My address can be found in the file ClassicThesis.pdf. A collection 
% of the postcards I received so far is available online at 
% http://postcards.miede.de
% ****************************************************************************************************

%Variable para controlar si es a color o en grises

\newif\ifColor
\Colortrue
%\Colorfalse

\newcommand{\colordepende}[2]
{\ifColor #1 \else #2 \fi }

% ****************************************************************************************************
% 0. Set the encoding of your files. UTF-8 is the only sensible encoding nowadays. If you can't read
% äöüßáéçèê∂åëæƒÏ€ then change the encoding setting in your editor, not the line below. If your editor
% does not support utf8 use another editor!
% ****************************************************************************************************
\PassOptionsToPackage{utf8}{inputenc}
	\usepackage[utf8]{inputenc}

% ****************************************************************************************************
% 1. Configure classicthesis for your needs here, e.g., remove "drafting" below 
% in order to deactivate the time-stamp on the pages
% ****************************************************************************************************
\PassOptionsToPackage{eulerchapternumbers,listings,drafting,%
					 pdfspacing,%linedheaders,%floatperchapter,%linedheaders,%
					 subfig,beramono,eulermath,parts}{classicthesis}                                        
% ********************************************************************
% Available options for classicthesis.sty 
% (see ClassicThesis.pdf for more information):
% drafting
% parts nochapters linedheaders
% eulerchapternumbers beramono eulermath pdfspacing minionprospacing
% tocaligned dottedtoc manychapters
% listings floatperchapter subfig
% ********************************************************************


% ****************************************************************************************************
% 2. Personal data and user ad-hoc commands
% ****************************************************************************************************
\newcommand{\myTitle}{Sistema Inteligente de Captaci\'on de Comunicaciones Inal\'ambricas para el An\'alisis y Predicci\'on de la Movilidad mediante Soft Computing\xspace}
\newcommand{\mySubtitle}{ \xspace}
\newcommand{\myDegree}{Doctor en Tecnolog\'ias de la Informaci\'on y la Comunicaci\'on\xspace}
\newcommand{\myName}{Antonio J. Fern\'andez Ares\xspace}
\newcommand{\myProf}{Pedro A. Castillo Valdivieso\xspace}
\newcommand{\myOtherProf}{Mar\'ia Isabel Garc\'ia Arenas\xspace}
\newcommand{\mySupervisor}{Put name here\xspace}
\newcommand{\myFaculty}{Escuela T\'ecnica Superior de Ingenier\'ias en Inform\'atica y Telecomunicaciones\xspace}
\newcommand{\myDepartment}{Departamento de Arquitectura y Tecnolog\'ia de los computadores\xspace}
\newcommand{\myUni}{TIC024 - Software Libre para Optimizaci\'on, B\'usqueda y Aprendizaje\xspace}
\newcommand{\myLocation}{Granada\xspace}
\newcommand{\myTime}{Junio 2019\xspace}
\newcommand{\myVersion}{version 0.9\xspace}

% ********************************************************************
% Setup, finetuning, and useful commands
% ********************************************************************
\newcounter{dummy} % necessary for correct hyperlinks (to index, bib, etc.)
\newlength{\abcd} % for ab..z string length calculation
\providecommand{\mLyX}{L\kern-.1667em\lower.25em\hbox{Y}\kern-.125emX\@}
\newcommand{\ie}{i.\,e.}
\newcommand{\Ie}{I.\,e.}
\newcommand{\eg}{e.\,g.}
\newcommand{\Eg}{E.\,g.} 
% ****************************************************************************************************


% ****************************************************************************************************
% 3. Loading some handy packages
% ****************************************************************************************************
% ******************************************************************** 
% Packages with options that might require adjustments
% ******************************************************************** 
%\PassOptionsToPackage{ngerman,american}{babel}   % change this to your language(s)
% Spanish languages need extra options in order to work with this template
%\PassOptionsToPackage{spanish,es-lcroman}{babel}
%	\usepackage{babel}   


\usepackage[spanish,es-lcroman,es-tabla,english]{babel}
          

%\usepackage{csquotes}
%\PassOptionsToPackage{%
%    %backend=biber, %instead of bibtex
%	backend=bibtex8,
%    bibencoding=ascii,%
%	language=auto,%
%	style=numeric-comp,%
%    %style=authoryear-comp, % Author 1999, 2010
%    %bibstyle=authoryear,dashed=false, % dashed: substitute rep. author with ---
%    sorting=nyt, % name, year, title
%    maxbibnames=10, % default: 3, et al.
%    %backref=true,%
%    natbib=true % natbib compatibility mode (\citep and \citet still work)
%}{biblatex}
%    \usepackage{biblatex}

\PassOptionsToPackage{fleqn}{amsmath}       % math environments and more by the AMS 
    \usepackage{amsmath}



\usepackage{csquotes}
\PassOptionsToPackage{%
    %backend=biber, %instead of bibtex
	backend=biber,
    bibencoding=utf8,%utf8
	language=auto,%
	style=numeric-comp,%
    %style=reading,%
    %citestyle=numeric-comp,
    %style=authoryear-comp, % Author 1999, 2010
    %bibstyle=authoryear,dashed=false, % dashed: substitute rep. author with ---
    sorting=nyt, % orden de la bibliografía name, year, title
    maxbibnames=10, % default: 3, et al.
    %backref=true,%
    natbib=true % natbib compatibility mode (\citep and \citet still work)
}{biblatex}
    \usepackage{biblatex}

\PassOptionsToPackage{fleqn}{amsmath}       % math environments and more by the AMS 
    \usepackage{amsmath}

% ******************************************************************** 
% General useful packages
% ******************************************************************** 




\PassOptionsToPackage{T1}{fontenc} % T2A for cyrillics
    \usepackage{fontenc}     
\usepackage{textcomp} % fix warning with missing font shapes
\usepackage{scrhack} % fix warnings when using KOMA with listings package          
\usepackage{xspace} % to get the spacing after macros right  
\usepackage{mparhack} % get marginpar right
\usepackage{fixltx2e} % fixes some LaTeX stuff --> since 2015 in the LaTeX kernel (see below)
%\usepackage[latest]{latexrelease} % will be used once available in more distributions (ISSUE #107)
\PassOptionsToPackage{printonlyused,smaller}{acronym} 
    \usepackage{acronym} % nice macros for handling all acronyms in the thesis
    %\renewcommand{\bflabel}[1]{{#1}\hfill} % fix the list of acronyms --> no longer working
    %\renewcommand*{\acsfont}[1]{\textsc{#1}} 
    %\renewcommand*{\aclabelfont}[1]{\acsfont{#1}}
    %\def\bflabel#1{{#1\hfill}}
    \def\bflabel#1{{\acsfont{#1}\hfill}}
    \def\aclabelfont#1{\acsfont{#1}}% ****************************************************************************************************


% ****************************************************************************************************
% 4. Setup floats: tables, (sub)figures, and captions
% ****************************************************************************************************
\usepackage{tabularx} % better tables
    \setlength{\extrarowheight}{3pt} % increase table row height
\newcommand{\tableheadline}[1]{\multicolumn{1}{c}{\spacedlowsmallcaps{#1}}}
\newcommand{\myfloatalign}{\centering} % to be used with each float for alignment
\usepackage{caption}
% Thanks to cgnieder and Claus Lahiri
% http://tex.stackexchange.com/questions/69349/spacedlowsmallcaps-in-caption-label
% [REMOVED DUE TO OTHER PROBLEMS, SEE ISSUE #82]    
%\DeclareCaptionLabelFormat{smallcaps}{\bothIfFirst{#1}{~}\MakeTextLowercase{\textsc{#2}}}
%\captionsetup{font=small,labelformat=smallcaps} % format=hang,
%\captionsetup{font=normalsize,justification=centering,singlelinecheck=false} % format=hang,
\captionsetup{font=normalsize,justification=centering,singlelinecheck=false} % format=hang,
\usepackage{subfig}
% ****************************************************************************************************


% ****************************************************************************************************
% 5. Setup code listings
% ****************************************************************************************************
%\usepackage{listings} 
%\lstset{emph={trueIndex,root},emphstyle=\color{BlueViolet}}%\underbar} % for special keywords
% \lstset{language=[LaTeX]Tex,%C++,
%     morekeywords={PassOptionsToPackage,selectlanguage},
%     keywordstyle=\color{RoyalBlue},%\bfseries,
%     basicstyle=\small\ttfamily,
%     %identifierstyle=\color{NavyBlue},
%     commentstyle=\color{Green}\ttfamily,
%     stringstyle=\rmfamily,
%     numbers=none,%left,%
%     numberstyle=\scriptsize,%\tiny
%     stepnumber=5,
%     numbersep=8pt,
%     showstringspaces=false,
%     breaklines=true,
%     %frameround=ftff,
%     %frame=single,
%     belowcaptionskip=.75\baselineskip
%     %frame=L
% } 

%ANTARES: Para añadir pseudocódigos

\usepackage[chapter]{algorithm}
\floatname{algorithm}{C\'odigo}
\usepackage{algpseudocode} 
\usepackage{minted}

\makeatletter
\let\algorithmname\fname@algorithm\space  

\setminted{
xleftmargin=5pt, %Puesto esto para arreglarlo dentro de los bloques de experimentos
linenos=true,
%frame=lines,
frame=leftline,
frame=single,
%xleftmargin=\parindent, % Quitado esto para arreglarlo dentro de los bloques de experimentos
framerule=1pt,
numberblanklines=true,
showspaces=false,
breaklines=true,
fontsize=\footnotesize}


% ****************************************************************************************************




% ****************************************************************************************************
% 6. PDFLaTeX, hyperreferences and citation backreferences
% ****************************************************************************************************
% ********************************************************************
% Using PDFLaTeX
% ********************************************************************
%\PassOptionsToPackage{hyphens}{url} %romper urls
\PassOptionsToPackage{pdftex,hyperfootnotes=true,pdfpagelabels}{hyperref}
    \usepackage{hyperref}  % backref linktocpage pagebackref %DESCOMENTAR PARA REFERENCIAS EN COLOR

%Enlace en los footnote
\usepackage[symbol=$\uparrow$]{footnotebackref}


%Para usar el resizebox y más cosas chulas
\pdfcompresslevel=9
\pdfadjustspacing=1 
\PassOptionsToPackage{pdftex}{graphicx}
    \usepackage{graphicx} 
 

% ********************************************************************
% Hyperreferences
% ********************************************************************
\hypersetup{%
    %draft, % = no hyperlinking at all (useful in b/w printouts)
    colorlinks=true, linktocpage=true, pdfstartpage=3, pdfstartview=FitV,%
    % uncomment the following line if you want to have black links (e.g., for printing)
    %colorlinks=false, linktocpage=false, pdfstartpage=3, pdfstartview=FitV, pdfborder={0 0 0},%
    breaklinks=true, pdfpagemode=UseNone, pageanchor=true, pdfpagemode=UseOutlines,%
    plainpages=false, bookmarksnumbered, bookmarksopen=true, bookmarksopenlevel=1,%
    hypertexnames=true, pdfhighlight=/O,%nesting=true,%frenchlinks,%
    urlcolor=webbrown, linkcolor=RoyalBlue, citecolor=webgreen, %pagecolor=RoyalBlue,%
    %urlcolor=Black, linkcolor=Black, citecolor=Black, %pagecolor=Black,%
    pdftitle={\myTitle},%
    pdfauthor={\textcopyright\ \myName, \myUni, \myFaculty},%
    pdfsubject={},%
    pdfkeywords={},%
    pdfcreator={pdfLaTeX},%
    pdfproducer={LaTeX with hyperref and classicthesis}%
}   

% ********************************************************************
% Setup autoreferences
% ********************************************************************
% There are some issues regarding autorefnames
% http://www.ureader.de/msg/136221647.aspx
% http://www.tex.ac.uk/cgi-bin/texfaq2html?label=latexwords
% you have to redefine the makros for the 
% language you use, e.g., american, ngerman
% (as chosen when loading babel/AtBeginDocument)
% ********************************************************************
\makeatletter
\@ifpackageloaded{babel}%
    {%
       \addto\extrasspanish{%
			\renewcommand*{\figureautorefname}{Figura}%
			\renewcommand*{\tableautorefname}{Tabla}%
			\renewcommand*{\partautorefname}{Parte}%
			\renewcommand*{\chapterautorefname}{Capítulo}%
			\renewcommand*{\sectionautorefname}{Secci\'on}%
			\renewcommand*{\subsectionautorefname}{Secci\'on}%
			\renewcommand*{\subsubsectionautorefname}{Secci\'on}%     
                }%
       \addto\extrasngerman{% 
			\renewcommand*{\paragraphautorefname}{Absatz}%
			\renewcommand*{\subparagraphautorefname}{Unterabsatz}%
			\renewcommand*{\footnoteautorefname}{Fu\"snote}%
			\renewcommand*{\FancyVerbLineautorefname}{Zeile}%
			\renewcommand*{\theoremautorefname}{Theorem}%
			\renewcommand*{\appendixautorefname}{Anhang}%
			\renewcommand*{\equationautorefname}{Gleichung}%        
			\renewcommand*{\itemautorefname}{Punkt}%
                }%  
            % Fix to getting autorefs for subfigures right (thanks to Belinda Vogt for changing the definition)
            \providecommand{\subfigureautorefname}{\figureautorefname}%             
    }{\relax}
\makeatother


% ****************************************************************************************************
% 7. Last calls before the bar closes
% ****************************************************************************************************
% ********************************************************************
% Development Stuff
% ********************************************************************
\listfiles
%\PassOptionsToPackage{l2tabu,orthodox,abort}{nag}
%   \usepackage{nag}
%\PassOptionsToPackage{warning, all}{onlyamsmath}
%   \usepackage{onlyamsmath}

% ********************************************************************
% Last, but not least...
% ********************************************************************
\usepackage{tesis} 
% ****************************************************************************************************


% ****************************************************************************************************
% 8. Further adjustments (experimental)
% ****************************************************************************************************
% ********************************************************************
% Changing the text area
% ********************************************************************
%\linespread{1.05} % a bit more for Palatino
%\areaset[current]{312pt}{761pt} % 686 (factor 2.2) + 33 head + 42 head \the\footskip
%\setlength{\marginparwidth}{7em}%
%\setlength{\marginparsep}{2em}%

% ********************************************************************
% Using different fonts
% ********************************************************************
%\usepackage[oldstylenums]{kpfonts} % oldstyle notextcomp
%\usepackage[osf]{libertine}
%\usepackage[light,condensed,math]{iwona}
%\renewcommand{\sfdefault}{iwona}
%\usepackage{lmodern} % <-- no osf support :-(
%\usepackage{cfr-lm} % 
%\usepackage[urw-garamond]{mathdesign} <-- no osf support :-(
%\usepackage[default,osfigures]{opensans} % scale=0.95 
%\usepackage[sfdefault]{FiraSans}
% ****************************************************************************************************


%********************************************************************
% Colores en color
%*******************************************************


%*******************************************************
%*******************************************************
%Colores para versión virtual:
% \definecolor[named]{ugrOrange}{rgb}{.945,.365,.165} % PANTONE 179: {R:241, G:93, B:42}
% \definecolor[named]{ugrGray}{rgb}{.463,.482,.494}
% \definecolor[named]{darkOrange}{rgb}{.8,.3,0}
%*******************************************************
%*******************************************************

%*******************************************************
%*******************************************************
%Colores para imprimir
%\definecolor[named]{ugrOrange}{rgb}{.463,.482,.494} % PANTONE 179: {R:241, G:93, B:42}
%\definecolor[named]{ugrGray}{rgb}{.463,.482,.494}
%\definecolor[named]{darkOrange}{rgb}{.263,.282,.294}

%\definecolor[named]{myred}{red}
%\definecolor[named]{mygreen}{green}
%\definecolor[named]{myblue}{blue}




\colordepende{
    \definecolor[named]{ugrOrange}{rgb}{.945,.365,.165} % PANTONE 179: {R:241, G:93, B:42}
    \definecolor[named]{ugrGray}{rgb}{.463,.482,.494}
    \definecolor[named]{darkOrange}{rgb}{.8,.3,0}
    }{
    \definecolor[named]{ugrOrange}{rgb}{.463,.482,.494} % PANTONE 179: {R:241, G:93, B:42}
    \definecolor[named]{ugrGray}{rgb}{.463,.482,.494}
    \definecolor[named]{darkOrange}{rgb}{.263,.282,.294}

    \usemintedstyle{bw} % Para que el resaltado de código emplee blanco y negro
    }



%*******************************************************
%*******************************************************


\definecolor[named]{Black}{gray}{0}

%\definecolor[named]{ultralightOrange}{rgb}{1,.94,.72}
%\definecolor[named]{highlightOrange}{rgb}{1,.86,.29}
%\definecolor[named]{lightOrange}{rgb}{1,.45,0}
%\definecolor[named]{darkOrange}{rgb}{.8,.3,0}

\definecolor[named]{minimumOrange}{rgb}{1,.976,.474}
%\definecolor[named]{ugrOrange}{rgb}{.945,.365,.165} % PANTONE 179: {R:241, G:93, B:42}
%\definecolor[named]{ugrOrange}{cmyk}{0,.309,.368,0} % PANTONE 179: {C:0, M:79, Y:94, K:0}
%\definecolor[named]{ugrGray}{rgb}{.463,.482,.494}
\definecolor[named]{Gray}{rgb}{.463,.482,.494}

%\definecolorseries{ugrOrangeSerie}{rgb}{last}{ugrOrange}{white}
%\resetcolorseries[10]{ugrOrangeSerie}

\colorlet{ultralightOrange}{ugrOrange!5!minimumOrange}
\colorlet{highlightOrange}{ugrOrange!25!minimumOrange}
\colorlet{lighterOrange}{ugrOrange!40!minimumOrange}
\colorlet{lightOrange}{ugrOrange!75!minimumOrange}

\definecolor{colorCorporativoMasSuave}{named}{ultralightOrange}
\definecolor{colorCorporativoSuave}{named}{highlightOrange}
\definecolor{colorCorporativoMedioSuave}{named}{lighterOrange}
\definecolor{colorCorporativoMedio}{named}{lightOrange}
\definecolor{colorCorporativo}{named}{ugrOrange}
\definecolor{colorCorporativoOscuro}{named}{darkOrange}
\definecolor{Maroon}{named}{colorCorporativoOscuro}

\definecolor{headColor}{named}{white}%{colorCorporativoMasSuave}
\definecolor{titleNumbercolor}{named}{ugrGray}
\definecolor{titlecolor}{named}{ugrOrange}
\definecolor{alternateTitlecolor}{named}{darkOrange}

\newcolumntype{A}{%
>{\color{black}\columncolor{colorCorporativoSuave}}%
p{0.95\textwidth}}

\newcolumntype{B}{%
>{\color{black}\columncolor{colorCorporativoMasSuave}}%
p{0.95\textwidth}}

\newcolumntype{C}{%
>{\columncolor{colorCorporativoSuave}}p{\textwidth}}
%>{\columncolor{colorCorporativo}}p{4pt}}

\newcolumntype{D}{%
>{\color{black}\fontfamily{pbk}\selectfont\columncolor{colorCorporativoSuave}}p{8em}}
%>{\columncolor{colorCorporativo}}p{4pt}


% %********************************************************************
% % Colores a imprimir
% %*******************************************************

% \definecolor[named]{Black}{gray}{0}

% %\definecolor[named]{ultralightOrange}{rgb}{1,.94,.72}
% %\definecolor[named]{highlightOrange}{rgb}{1,.86,.29}
% %\definecolor[named]{lightOrange}{rgb}{1,.45,0}
% \definecolor[named]{darkOrange}{rgb}{.8,.3,0}

% \definecolor[named]{minimumOrange}{rgb}{1,.976,.474}
% \definecolor[named]{ugrOrange}{rgb}{.945,.365,.165} % PANTONE 179: {R:241, G:93, B:42}
% %\definecolor[named]{ugrOrange}{cmyk}{0,.309,.368,0} % PANTONE 179: {C:0, M:79, Y:94, K:0}
% \definecolor[named]{ugrGray}{rgb}{.463,.482,.494}
% \definecolor[named]{Gray}{rgb}{.463,.482,.494}

% %\definecolorseries{ugrOrangeSerie}{rgb}{last}{ugrOrange}{white}
% %\resetcolorseries[10]{ugrOrangeSerie}

% \colorlet{ultralightOrange}{ugrOrange!5!minimumOrange}
% \colorlet{highlightOrange}{ugrOrange!25!minimumOrange}
% \colorlet{lighterOrange}{ugrOrange!40!minimumOrange}
% \colorlet{lightOrange}{ugrOrange!75!minimumOrange}

% \definecolor{colorCorporativoMasSuave}{named}{ultralightOrange}
% \definecolor{colorCorporativoSuave}{named}{highlightOrange}
% \definecolor{colorCorporativoMedioSuave}{named}{lighterOrange}
% \definecolor{colorCorporativoMedio}{named}{lightOrange}
% \definecolor{colorCorporativo}{named}{ugrOrange}
% \definecolor{colorCorporativoOscuro}{named}{darkOrange}
% \definecolor{Maroon}{named}{colorCorporativoOscuro}

% \definecolor{headColor}{named}{white}%{colorCorporativoMasSuave}
% \definecolor{titleNumbercolor}{named}{ugrGray}
% \definecolor{titlecolor}{named}{ugrOrange}
% \definecolor{alternateTitlecolor}{named}{darkOrange}

% \newcolumntype{A}{%
% >{\color{black}\columncolor{colorCorporativoSuave}}%
% p{0.95\textwidth}}

% \newcolumntype{B}{%
% >{\color{black}\columncolor{colorCorporativoMasSuave}}%
% p{0.95\textwidth}}

% \newcolumntype{C}{%
% >{\columncolor{colorCorporativoSuave}}p{\textwidth}}
% %>{\columncolor{colorCorporativo}}p{4pt}}

% \newcolumntype{D}{%
% >{\color{black}\fontfamily{pbk}\selectfont\columncolor{colorCorporativoSuave}}p{8em}}
% %>{\columncolor{colorCorporativo}}p{4pt}





%***************
% Minitoc
%***************

\usepackage{minitoc}
\mtcsettitle{minitoc}{\'Indice del cap\'itulo}

\setcounter{minitocdepth}{1}

% ***********************************
% Parámetros para Lettrine (lo de la primera letra más grande en los comienzos de los capítulos)
% **************************

\usepackage{lettrine}


\setcounter{DefaultLines}{2}
\renewcommand{\DefaultLoversize}{0.1} %0.25
\renewcommand{\DefaultLraise}{0.3} %0.25
\renewcommand{\DefaultLhang}{0.5} %0.5
\setlength{\DefaultFindent}{0pt}%0.5em}
\setlength{\DefaultNindent}{0pt}%0.1em}
\setlength{\DefaultSlope}{0pt}
\renewcommand{\LettrineFontHook}{\color{colorCorporativoOscuro}\fontfamily{fau}}%{fau}pzc}\fontseries{bx}\fontshape{it}}
\renewcommand{\LettrineTextFont}{\color{colorCorporativoOscuro}\fontfamily{fau}\scshape}


\usepackage{changepage}


%Para las multirows
\usepackage{multirow}

%Espacio detrás del párrafo
\parskip = 6pt

%Para hacer diagramas dentro del propio LATEX
\usepackage{graphicx}
\usepackage[all]{xy}
\usepackage{tikz}

\usetikzlibrary{arrows,shadows,snakes}
%\usetikzlibrary{arrows.meta}
\usetikzlibrary{shapes,arrows,intersections,backgrounds}
\usetikzlibrary{matrix,fit,calc,trees,positioning,arrows,chains,shapes.geometric,shapes}
\usetikzlibrary{intersections}
\usetikzlibrary{positioning}

%Para poder hacer gráficas de conjuntos de datos
\usepackage{pgfplots}
\pgfdeclarelayer{background}
%\pgfsetlayers{background}

\pgfkeys{/pgf/number format/.cd,1000 sep={\,}}

\definecolor{butter1}{rgb}{0.988,0.914,0.310}
\definecolor{chocolate1}{rgb}{0.914,0.725,0.431}
\definecolor{chameleon1}{rgb}{0.541,0.886,0.204}
\definecolor{skyblue1}{rgb}{0.447,0.624,0.812}
\definecolor{plum1}{rgb}{0.678,0.498,0.659}
\definecolor{scarletred1}{rgb}{0.937,0.161,0.161}

\pgfplotscreateplotcyclelist{miscolores}{
 {butter1},{chocolate1},{chameleon1},{skyblue1},{plum1},{scarletred1}}

%Para hacer gráficas de tarta
%\usepackage{pgf-pie}

%Para arreglar el error del resizebox en el tikz

\tikzset{
every picture/.append style={
  execute at begin picture={\deactivatequoting},
  execute at end picture={\activatequoting}
  }
}
\AtBeginEnvironment{tikzpicture}{\shorthandoff{>}\shorthandoff{<}}{}{}

\deactivatequoting


% ********************************************************************
% Para el aspecto gráfico molón
%*******************************************************
\DeclareCaptionFont{titlecolor}{\color{titlecolor}\fontfamily{pbk}\selectfont}
\DeclareCaptionFont{alternateTitlecolor}{\color{alternateTitlecolor}\fontfamily{pbk}\selectfont}
\captionsetup{justification=justified,format=plain,labelsep=newline,font=scriptsize,labelfont=titlecolor,textfont=alternateTitlecolor}%
\captionsetup[algorithm]{justification=justified,format=plain,labelsep=newline,font=scriptsize,labelfont=titlecolor,textfont=alternateTitlecolor}%
% ********************************************************************
% Where to look for graphics
%*******************************************************
%\graphicspath{{gfx/}{misc/}} % considered harmful according to l2tabu
% ********************************************************************
% Hyperreferences
%*******************************************************
\hypersetup{%
    colorlinks=true, linktocpage=true, pdfstartpage=3, pdfstartview=FitV,%
    breaklinks=true, pdfpagemode=UseNone, pageanchor=true, pdfpagemode=UseOutlines,%
    plainpages=false, bookmarksnumbered, bookmarksopen=true, bookmarksopenlevel=1,%
    hypertexnames=true, pdfhighlight=/O,%hyperfootnotes=true,%nesting=true,%frenchlinks,%
    urlcolor=colorCorporativoOscuro, linkcolor=alternateTitlecolor, citecolor=Maroon, %pagecolor=RoyalBlue,%
    % uncomment the following line if you want to have black links (e.g., for printing)
    %urlcolor=Black, linkcolor=Black, citecolor=Black, %pagecolor=Black,%
    %urlcolor=Gray, linkcolor=Gray, citecolor=Gray, %pagecolor=Black,%
    pdftitle={\myTitle},%
    pdfauthor={\textcopyright\ \myName, \myUni, \myFaculty},%
    pdfsubject={},%
    pdfkeywords={},%
    pdfcreator={pdfLaTeX},%
    pdfproducer={LaTeX with hyperref and classicthesis}%
}

% Parámetros para Lettrine (lo de la primera letra más grande en los comienzos de los capítulos)
%
\setcounter{DefaultLines}{2}
\renewcommand{\DefaultLoversize}{0.1} %0.25
\renewcommand{\DefaultLraise}{0.3} %0.25
\renewcommand{\DefaultLhang}{0.5} %0.5
\setlength{\DefaultFindent}{0pt}%0.5em}
\setlength{\DefaultNindent}{0pt}%0.1em}
\setlength{\DefaultSlope}{0pt}
\renewcommand{\LettrineFontHook}{\color{colorCorporativoOscuro}\fontfamily{fau}}%{fau}pzc}\fontseries{bx}\fontshape{it}}
\renewcommand{\LettrineTextFont}{\color{colorCorporativoOscuro}\fontfamily{fau}\scshape}

\pagenumbering{roman}
\pagestyle{scrheadings}%{plain}

\renewcommand{\labelenumii}{\arabic{enumii}.}



%COLORES DE LOS DIAGRAMAS



%PARA REFERENCIAR SUBFIGURAS:
\captionsetup[subfigure]{subrefformat=simple,labelformat=simple,listofformat=subsimple}
\renewcommand\thesubfigure{(\alph{subfigure})}
\captionsetup[subfigure]{justification=centering}

%DEFINICIONES
\newtheorem{SmartCitydef}{Definici\'on}

%Para las tablas con lineas diagonales
\usepackage{diagbox}

%PARA QUE SALGAN TAMBIÉN LOS PARRAFOS EN EL INDICE:
% \setcounter{secnumdepth}{4}
% \cftsetindents{section}{0.3in}{0.3in}
% \cftsetindents{subsection}{0.4in}{0.4in}
% \cftsetindents{subsubsection}{0.5in}{0.5in}
% \cftsetindents{paragraph}{0.6in}{0.6in}

\setcounter{secnumdepth}{4}
\cftsetindents{section}{0.4in}{0.4in}
\cftsetindents{subsection}{0.5in}{0.5in}
\cftsetindents{subsubsection}{0.6in}{0.6in}
\cftsetindents{paragraph}{0.7in}{0.7in}

%PARA HACER DIAGRAMAS DE BITS
\usepackage{bytefield}
\RequirePackage[numberFieldsOnce,bigEndian,numberBitsAbove]{bitpattern}
\renewcommand\bpFormatBitNumber[1]{{\tiny\ttfamily\emph{\strut#1}}}
\renewcommand\bpFormatField[1]{{\scriptsize\ttfamily\strut#1}}

%PARA QUE SALGAN LA FIGURAS NUMERADAS COMO CAPITULO MÁS NÚMERO
\numberwithin{figure}{chapter}
\numberwithin{table}{chapter}
\numberwithin{equation}{chapter}
\numberwithin{algorithm}{chapter}

%PARA QUE LAS TABLAS SALGAN DONDE TIENEN QUE SALIR, NO SE VAYAN A OTRA SECCIÓN
\usepackage[section]{placeins}


%Para tener glosarios:
%\usepackage[style=long,nolist]{glossaries}
%\usepackage[xindy]{glossaries}
%\usepackage[style=long,nolist]{glossaries}
%\usepackage[style=long, nonumberlist, toc, xindy, nowarn, nomain, section=chapter]{glossaries}
%\usepackage[style=long, toc, xindy, section=chapter]{glossaries}
%\usepackage[style=indexhypergroup, toc, xindy, section=chapter]{glossaries}
\usepackage[style=treegroup, toc, xindy, section=chapter]{glossaries}

\newglossarystyle{mytreegroup}{%
  \glossarystyle{treegroup}%
  \renewenvironment{theglossary}%
    {\setlength{\parindent}{0pt}%
     \setlength{\parskip}{0pt plus 0.3pt}}%
    {}%
  \renewcommand*{\glossaryheader}{}%
  \renewcommand*{\glsgroupheading}[1]{}%
  \renewcommand{\glossaryentryfield}[5]{%
    \hangindent0pt\relax
    \parindent0pt\relax
    %\glsentryitem{##1}\textbf{\glstarget{##1}{##2}}%
    \glsentryitem{##1}\glstarget{##1}{##2}%
    \ifx\relax##4\relax
    \else
      \space(##4)%
    \fi %\dotfill
    \dotfill ##5 \\ \space ##3\glspostdescription \\\par }%
  \renewcommand{\glossarysubentryfield}[6]{%
    \hangindent##1\glstreeindent\relax
    \parindent##1\glstreeindent\relax
    \ifnum##1=1\relax
      \glssubentryitem{##2}%
    \fi
    \textbf{\glstarget{##2}{##3}}%
    \ifx\relax##5\relax
    \else
      \space(##5)%
    \fi
    \space##4\glspostdescription\space ##6\par}%
  \renewcommand*{\glsgroupskip}{\indexspace}
  \renewcommand{\glsgroupheading}[1]{\par
    \noindent\textbf{\glsgetgrouptitle{##1}}\par\indexspace}%
}
\glossarystyle{mytreegroup}

%\usepackage[style=long,nolist]{glossaries}
%Cargamos el glosario

\newglossaryentry{example}{
name=example,
plural=example,
%first={},
description={example}
}
\newglossaryentry{iot}{
name=iot,
plural=iot,
%first={},
description={Internet of Things}
}



% Para los símbolos
\usepackage{pifont}% http://ctan.org/pkg/pifont
\newcommand{\cmark}{\ding{51}}%
\newcommand{\xmark}{\ding{55}}%

% Para añadir relojes
\usepackage{xparse}

\NewDocumentCommand{\Clock}{O{2cm}O{\large}O{cyan}O{0}O{90}O{120}O{270}}{%
%\def\radius{ (#1-3)*(-30)   }%
\def\radius{#1}%
\def\thora{#4}%
\def\tmin{#5}%
\def\tseg{#6}%

\begin{tikzpicture}[line cap=rect,line width=0.055*\radius]
\filldraw [fill=#3] (0,0) circle [radius=\radius];
\foreach \angle [count=\xi] in {60,30,...,-270}
{
  \draw[line width=1pt] (\angle:0.9*\radius) -- (\angle:\radius);
  \node[font=#2] at (\angle:0.68*\radius) {\textsf{\xi}};
}
\foreach \angle in {0,90,180,270}
  \draw[line width=0.04*\radius] (\angle:0.82*\radius) -- (\angle:\radius);

\draw (0,0) -- ({(\tmin-15)*(-6)}:0.5*\radius);
\draw [black!50,line width=2pt] (0,0) -- ({(\thora-3)*(-30)}:0.3*\radius);
\draw [red, line width=1pt] (0,0) -- ({(\tseg-15)*(-6)}:0.7*\radius);
\end{tikzpicture}%
}

%Para aumentar el número de writers
%\usepackage{morewrites}


%Para añadir las tablas de los experimentos
\NewDocumentCommand{\tablaExperimento}{O{Escenario}O{Monitorizaci\'on}O{Experimentos relacionados}O{Ambito}O{Publicado}O{Rango de Fechas}O{Dataset}O{Objetivos}}{%
\vspace{-0.5cm}
\hspace{-0.1cm}
\resizebox{1\textwidth}{!}{%
\begin{tikzpicture}[line cap=rect,line width=0.5]

%\draw [help lines] (0,0) grid (16,6);

\def\columnalarga{19.8cm}%
\def\columna{9.8cm}%
\def\mediacolumna{4.8cm}%
\def\altocolumna{0.8cm}%
\def\distanciatitulo{-0.5cm}%
\def\distanciatitulobajo{-0.6cm}%

\node[draw, text depth=0cm,rounded corners,align=center,text width=4cm, label={[gray,yshift=\distanciatitulobajo]above:{\emph{Número de experimento}}}, minimum width = \mediacolumna, minimum height = \altocolumna] at (2.5,5.5){\rule{0pt}{0.65cm} \LARGE \emph{\arabic{experimentos}} };

\node[draw, text depth=0cm,rounded corners,align=center,text width=4cm, label={[gray,yshift=\distanciatitulo]above:{\emph{Escenario}}}, minimum width = \mediacolumna, minimum height = \altocolumna] at (7.5,5.5){\rule{0pt}{0.65cm} #1 };

\node[draw, text depth=0cm,rounded corners,align=center,text width=4cm, label={[gray,yshift=\distanciatitulo]above:{\emph{Monitorizaci\'on}}}, minimum width = \mediacolumna, minimum height = \altocolumna] at (12.5,5.5){\rule{0pt}{0.65cm} #2 };

\node[draw, text depth=0cm,rounded corners,align=center,text width=4cm, label={[gray,yshift=\distanciatitulobajo]above:{\emph{Relacionados}}}, minimum width = \mediacolumna, minimum height = \altocolumna] at (17.5,5.5){\rule{0pt}{0.65cm} #3 };

\node[draw, text depth=0cm,rounded corners,align=center,text width=12cm, label={[gray,yshift=\distanciatitulo]above:{\emph{\'Ambito}}}, minimum width = \columnalarga, minimum height = \altocolumna] at (10,4.5){\rule{0pt}{0.65cm} #4 };

%\node[draw, text depth=0cm,rounded corners,align=center,text width=9cm, label={[gray,yshift=\distanciatitulo]above:{\emph{\'Publicado en}}}, minimum width = \columna, minimum height = \altocolumna] at (15,4.5){\rule{0pt}{0.65cm} #5 };

\node[draw, text depth=0cm,rounded corners,align=center,text width=9cm, label={[gray,yshift=\distanciatitulobajo]above:{\emph{Rango de fechas conjunto de datos}}}, minimum width = \columna, minimum height = \altocolumna] at (5,3.5){\rule{0pt}{0.65cm} #6 };

\node[draw, text depth=0cm,rounded corners,align=center,text width=9cm, label={[gray,yshift=\distanciatitulobajo]above:{\emph{Características del conjunto de datos}}}, minimum width = \columna, minimum height = \altocolumna] at (15,3.5){\rule{0pt}{0.65cm} #7 };

\node[draw, text depth=0cm,rounded corners,align=center,text width=12cm, label={[gray,yshift=\distanciatitulobajo]above:{\emph{\'Objetivos}}}, minimum width = \columnalarga, minimum height = \altocolumna] at (10,2.5){\rule{0pt}{0.65cm} #8 };
\end{tikzpicture}
}%
}

%Para poder referenciar los experimentos sin que aparezca el número de capítulo
%\renewcommand{\thesubsection}{\arabic{subsection}}


%Para añadir diagramas UML
\usepackage{pgf-umlsd}

%Notación:
% Hebras: 


%Para poder reajustar el tamaño de las imágenes tikz

\newsavebox\mybox
\newenvironment{resizedtikzpicture}[1]{%
  \def\mywidth{#1}%
  \begin{lrbox}{\mybox}%
  \begin{tikzpicture}
}{%
  \end{tikzpicture}%
  \end{lrbox}%
  \resizebox{\mywidth}{!}{\usebox\mybox}%
}


%Arreglar error nocite
\DeclareUnicodeCharacter{00A0}{ }

%Para que no separe palabras por guiones:
%\usepackage[none]{hyphenat}

%Para añadir árboles de directorios
\usepackage{dirtree}
%\DTsetlength{0.5em}{1em}{0.2em}{0.4pt}{3.6pt}

%Iconos


\newcommand\iconFOLDER{\colordepende{\includegraphics[height=2.5ex]{gfx/Recursos/ficheros/folder.pdf}}{\includegraphics[height=2.5ex]{gfx/Recursos/ficheros/gris/folder.pdf}} \hspace*{0.01ex}}
\newcommand\iconFILE{\includegraphics[height=2.5ex]{gfx/Recursos/ficheros/file.pdf}
 \hspace*{0.01ex}}
\newcommand\iconSD{\includegraphics[height=2.5ex]{gfx/Recursos/ficheros/sd.pdf}
 \hspace*{0.01ex}}

%Solventa el problema de "no room for a new \write"
\usepackage{morewrites}

%Emplea como sepador decimal un punto, en lugar de una coma
\spanishdecimal{.}

%No parte palabras en los títulos NO FUNCIONA

%\usepackage[raggedright]{titlesec}

%
% Diagramas ER
%

%\usepackage{tikz-er2}
% Include tikz library for more control over positioning
%\usetikzlibrary{positioning}
% Styling for entities, attributes, and relationships
%\tikzstyle{every entity} = [draw=blue!50!black!100, fill=blue!20]
%\tikzstyle{every attribute} = [draw=yellow!50!black!100, fill=yellow!20]
%\tikzstyle{every relationship} = [draw=green!50!black!100, fill=green!20]



%Añadir un listado de experimentos al índice

\usepackage{tocloft}

\newcommand{\listExperimentosname}{\'Indice de estudios}

\newlistof{Experimentos}{exp}{\listExperimentosname}

\newcounter{experimentos}
%Hace que el contador sea dependiente del capítulo
%\numberwithin{experimentos}{chapter}
\numberwithin{experimentos}{section}
%\numberwithin{experimentos}{subsection}

%\addtolength{\cftExperimentosnumwidth}{15pt}
%\cftsetindents{Experimentos}{0.7in}{0.7in}
\cftsetindents{Experimentos}{0cm}{0cm}
%\setlength{\cftExperimentosnumwidth}{1.1cm}
%\renewcommand{\cftExperimentosaftersnumb}{\hspace{1.5em}}
 

%\setlength\cftExperimentosindent{1.5em}
%\renewcommand{\cftExperimentosafterpnum}{\cftparfillskip}


%\setlength{\cftbeforechapskip}{0.3cm}

% \setcounter{secnumdepth}{4}
% \cftsetindents{section}{0.4in}{0.4in}
% \cftsetindents{subsection}{0.5in}{0.5in}
% \cftsetindents{subsubsection}{0.6in}{0.6in}
% \cftsetindents{paragraph}{0.7in}{0.7in}


%  \renewcommand{\cftExperimentospresnum}{\scshape\MakeTextLowercase}% 
%          \renewcommand{\cftExperimentosfont}{\normalfont}%                 
%        \ifthenelse{\boolean{@dottedtoc}}{\relax}%
%        {%
%            \renewcommand{\cftExperimentosleader}{\hspace{1.5em}}% 
%            \renewcommand{\cftExperimentosafterpnum}{\cftparfillskip}%
%        }
%        \renewcommand{\cftExperimentospresnum}{\figurename~}%Fig.~}
%        \newlength{\Experimentoslabelwidth}
%        \settowidth{\Experimentoslabelwidth}{\cftfigpresnum~999}
%        \addtolength{\Experimentoslabelwidth}{2.5em}
%        %\cftsetindents{Experimentos}{0em}{15pt}
% %\cftsetrmarg{15pt}

%\renewcommand*{\listtablename}{\hfill\bfseries\normalsize\MakeUppercase{Lista de Tabelas}\hfill}
% Not this:
%\renewcommand{\cfttabfont}{Tabela }
% But this:
%\renewcommand{\cfttabpresnum}{Tabela~}
%\renewcommand{\cfttabaftersnum}{ ---}
%\setlength{\cfttabindent}{0pt}
% Both \cfttabindent and \cfttabnumwidth need to be changed
%\cftsetindents{table}{0pt}{5.5em}% Adjust 5.5em if necessary

%\newcommand{\Experimento}[1]{%
%\refstepcounter{experimentos}
%Estudio \thechapter.\arabic{experimentos}: #1
%\addcontentsline{exp}{Experimentos}
%{\protect Estudio \numberline{\thechapter.\arabic{experimentos}}#1}}

\newcommand{\Experimento}[1]{BORRAR}

%Bordes al rededor de los experimentos

\makeatletter
\newcommand*{\currentname}{\@currentlabelname}
\makeatother



\usepackage{tcolorbox}
\tcbuselibrary{breakable}
\tcbuselibrary{skins}

%black!6!white

\newenvironment{experimento}[1][--]
{\refstepcounter{experimentos}\begin{tcolorbox}[breakable, oversize, colback=black!4!white, parbox=false,title={Estudio \thesection.\arabic{experimentos}: #1}]
\addcontentsline{exp}{Experimentos}
{\protect Estudio \numberline{\thesection.\arabic{experimentos}}\\ {\scriptsize \currentname} \\#1}}
{\end{tcolorbox}}

\newenvironment{experimentoNoframe}[1][--]
{\refstepcounter{experimentos}\begin{tcolorbox}[skin=enhancedfirst jigsaw, oversize,colback=white,parbox=false,title={{\scriptsize \currentname}\\Estudio \thesection.\arabic{experimentos}: #1}]
\addcontentsline{exp}{Experimentos}
{\protect Estudio \numberline{\thesection.\arabic{experimentos}} \\ {\scriptsize \currentname} \\#1}\end{tcolorbox}\vspace{-0.85cm}}
{\FloatBarrier\vspace{-0.85cm}\begin{tcolorbox}[skin=enhancedlast jigsaw, oversize, opacityback=0,colback=white,parbox=false]\end{tcolorbox}}

\newtcolorbox{marker}[1][]{enhanced,
  before skip=2mm,after skip=3mm,
  boxrule=0.4pt,left=5mm,right=2mm,top=1mm,bottom=1mm,
  colback=yellow!50,
  colframe=yellow!20!black,
  sharp corners,rounded corners=southeast,arc is angular,arc=3mm,
  underlay={%
    \path[fill=tcbcol@back!80!black] ([yshift=3mm]interior.south east)--++(-0.4,-0.1)--++(0.1,-0.2);
    \path[draw=tcbcol@frame,shorten <=-0.05mm,shorten >=-0.05mm] ([yshift=3mm]interior.south east)--++(-0.4,-0.1)--++(0.1,-0.2);
    \path[fill=yellow!50!black,draw=none] (interior.south west) rectangle node[white]{\Huge\bfseries !} ([xshift=4mm]interior.north west);
    },
  drop fuzzy shadow,#1}


% ********************************************************************
% Para el aspecto gráfico molón
%*******************************************************
\DeclareCaptionFont{titlecolor}{\color{titlecolor}\fontfamily{pbk}\selectfont}
\DeclareCaptionFont{alternateTitlecolor}{\color{alternateTitlecolor}\fontfamily{pbk}\selectfont}
\captionsetup{justification=justified,format=plain,labelsep=newline,font=scriptsize,labelfont=titlecolor,textfont=alternateTitlecolor}%

\newcommand\conclusiones{\tcblower {\color{titlecolor} Conclusiones}
}

\newcommand{\micite}[1]{\cite{#1}{\tiny\graffito{\cite{#1} \citefield{#1}{title}}}}
\newcommand{\miciteJMN}[1]{{\tiny\graffito{\cite{#1} \citefield{#1}{title}}}}


\newcommand{\gracias}[1]{\begin{flushleft}\titlerule\emph{#1:}\end{flushleft}\vspace{-1em}}



%Para añadir las conclusiones:


\newcounter{conclusiones}[section]

%\numberwithin{conclusiones}{section}

\newenvironment{conclusion}[1][]{\refstepcounter{conclusiones}\par\medskip
   \textbf{Conclusi\'on~\thesection.\theconclusiones. #1} \rmfamily}{\medskip}
 
 



